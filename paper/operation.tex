\section{Operation of the System}

% What does your demonstration do and how does it work? What does it look like?

Big Brother has two main components: a high-throughput server and a JavaScript client. The client listens to events that occur in a web browser, and the server ingests them in real-time.

\subsection{Client}

The Big Brother client has been written to provide maximum functionality with minimal setup and configuration. The smallest amount of code to have Big Brother listen to events is presented in Listing~\ref{lst:minimal}. By default, Big Brother is configured to listen to all events on all elements on a page. This client has been put to the test on multiple user studies with highly complex interfaces~\cite{todo}. Practitioners of these studies and users of their systems noticed no slowdown of their study as a result of using Big Brother.

\begin{lstlisting}[language=JavaScript, label=lst:minimal, caption=Minimal example of code required to initialise \bb on the client-side. Note the \texttt{session\_id} which should already be initialied elsewhere.]
<script src="bigbro.js"></script>
<script type="text/javascript">
BigBro.init(session_id, "localhost:1984");
</script>
\end{lstlisting}

Often, however, the practitioner of a user study may only be interested in a subset of events. Listing~\ref{lst:events} presents the JavaScript code required on the client-side to restrict Big Brother to listen to two events: \texttt{mousemove} and \texttt{onload}, which will record the positional information of the users' mouse, and will log the time the session began respectively.

\begin{lstlisting}[language=JavaScript, label=lst:events, caption=Initialising \bb to listen on only certain global events. HTML code removed for brevity.]
BigBro.init(session_id, 
            "localhost:1984", 
            ["mousemove", "onload"]);
\end{lstlisting}

Finally, sometimes, the practitioner of a user study may wish to capture custom interactions or events from their user. For example, logging a custom event for when a user clicks a specific button on the page. Listing~\ref{lst:custom} presents the JavaScript code required on the client-side to wire the processing of a custom event to Big Brother's exposed custom logging functionality.

\begin{lstlisting}[language=JavaScript, label=lst:custom, caption=Wiring up \bb to listen to click events and to log a custom event.]
let bb = BigBro.init(session_id, 
                     "localhost:1984");
let w = window;
w.addEventListener("click", function (e) {
    bb.log(e, "custom_event");
})
\end{lstlisting}

\subsection{Server}

\begin{table}
\centering
\begin{tabular}{p{0.2\linewidth}p{0.7\linewidth}}
\hline
Component & Description of Component \\
\hline
Target & The HTML element that has caused the event to trigger. \\
Name & The \texttt{name} attribute of the HTML element in Target. \\
ID & The \texttt{id} attribute of the HTML element in Target\\
Method & The name of the method that caused the event to trigger. \\
Location & The web-page location on the server (URL, with query string and anchors). \\
Comment & Any additional custom information that may be useful to interpreting the event. \\
X & The x-position within the web browser that the event occurred at. \\
Y & The y-position within the web browser that the event occurred at. \\
ScreenWidth & The width of the web browser the event occurred within. \\
ScreenHeight & The height of the web browser the event occurred within. \\
Time & The time the event happened.\\
Actor & An identifier that can be used to refer to the user that caused the event, e.g., session ID. \\
\hline
\end{tabular}
\caption{The components of an event.}
\label{tbl:event}
\end{table}

The server processes interactions from users as \textit{events}. The components that comprise an event are summarised in Table~\ref{tbl:event}.
An event captures information about the HTML elements that were interacted with (\textbf{Target}, \textbf{Name}, \textbf{ID}), how the event was triggered (\textbf{Method}), what the URL was on the page that the event occurred (\textbf{Location}), additional custom comments (\textbf{Comment}), positional information to "replay" interactions (\textbf{X}, \textbf{Y}, \textbf{ScreenWidth}, \textbf{ScreenHeight}), the time that the event occurred (\textbf{Time}), and the user that caused the event (\textbf{Actor}).

Events are streamed from the user's web browser to a centralised server over websockets. This enables the real-time ingestion of interactions. Currently, Big Brother can output logs to a local csv file as well as directly indexing them to Elasticsearch. A small, real example of logs captured during a user study is presented in Listing~\ref{lst:logs}. The researcher who ran this study decided to output their interaction logs directly to a csv file. The ordering of components in these events is how they are produced by \bb. Listing~\ref{lst:components} shows the ordering of components in the csv file output. These logs show the interactions of two user sessions from a human intelligence task (HIT) on a popular crowdsourcing platform. This slice of interaction logs begins when the \texttt{A1} user starts a HIT, and ends when they finish the HIT. At the same time, another user (\texttt{A2}) is in the middle of another HIT. The researcher running this user study records when their users click on a multiple select dropdown. The very final line in the log, when user \texttt{A1} finishes the HIT, also logs the recorded answers to the questionnaire using the Comment component: \texttt{T|T|T|T|T}.

\begin{figure*}
\begin{lstlisting}[caption=Order of event components in Listing~\ref{lst:logs},label=lst:components]
Time,Actor,Method,Target,Name,ID,Location,X,Y,ScreenWidth,ScreenHeight,Comment
\end{lstlisting}
\begin{lstlisting}[basicstyle=\footnotesize\ttfamily,caption=Example log file capturing two user sessions at the same time. Note that there were far more interactions captured during the study and that this is a small contrived slice of the data. Much of the information from the URL has been redacted.,label=lst:logs]
19-01-03 06:12:45,A1,hit_start,,,,https://example.com?assignmentId=XYZ123&hitId=HIT1,0,0,1319,646,
19-01-03 06:12:47,A1,click,SELECT,q1,,https://example.com?assignmentId=XYZ123&hitId=HIT1,1079,299,1319,646,
19-01-03 06:12:48,A1,click,SELECT,q1,,https://example.com?assignmentId=XYZ123&hitId=HIT1,0,0,1319,646,
19-01-03 06:12:49,A1,click,SELECT,q2,,https://example.com?assignmentId=XYZ123&hitId=HIT1,1067,328,1319,646,
19-01-03 06:12:46,A2,click,SELECT,q4,,https://example.com?assignmentId=ABC789&hitId=HIT2,806,233,1063,306,
19-01-03 06:12:46,A2,click,SELECT,q4,,https://example.com?assignmentId=ABC789&hitId=HIT2,0,0,1063,306,
19-01-03 06:12:50,A1,click,SELECT,q2,,https://example.com?assignmentId=XYZ123&hitId=HIT1,0,0,1319,646,
19-01-03 06:12:50,A1,click,SELECT,q3,,https://example.com?assignmentId=XYZ123&hitId=HIT1,1067,354,1319,646,
19-01-03 06:12:51,A1,click,SELECT,q3,,https://example.com?assignmentId=XYZ123&hitId=HIT1,0,0,1319,646,
19-01-03 06:12:49,A2,click,SELECT,q5,,https://example.com?assignmentId=ABC789&hitId=HIT2,779,260,1063,306,
19-01-03 06:12:52,A1,click,SELECT,q4,,https://example.com?assignmentId=XYZ123&hitId=HIT1,1069,390,1319,646,
19-01-03 06:12:50,A2,click,SELECT,q5,,https://example.com?assignmentId=ABC789&hitId=HIT2,0,0,1063,306,
19-01-03 06:12:53,A1,click,SELECT,q4,,https://example.com?assignmentId=XYZ123&hitId=HIT1,0,0,1319,646,
19-01-03 06:12:54,A1,click,SELECT,q5,,https://example.com?assignmentId=XYZ123&hitId=HIT1,1074,424,1319,646,
19-01-03 06:12:55,A1,click,SELECT,q5,,https://example.com?assignmentId=XYZ123&hitId=HIT1,0,0,1319,646,
19-01-03 06:12:56,A1,hit_end,INPUT,,submit,https://example.com?assignmentId=XYZ123&hitId=HIT1,659,617,1319,646,T|T|T|T|T
\end{lstlisting}
\end{figure*}



