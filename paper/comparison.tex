\vspace{-16pt}
\section{Comparison with Existing Systems}

The idea to capture interactions in user study scenarios is not new. For example, Zuccon et al.~\cite{zuccon2013crowdsourcing} proposed a system for evaluating Information Retrieval systems using interactions from crowdsourced workers. They observe interactions such as the length of sessions, the number of elements clicked, and the number of items examined. However, they do not measure fine-grained interactions such as mouse movements, and these measurements are built into the underlying interface, unlike Big Brother which can log interactions independently of the underlying interface. Bierig et al.~\cite{bierig2009user} proposed an end-to-end system for designing user studies and logging multi-modal interactions within those studies. The downside to this system is that it may not provide the flexibility afforded by bespoke user study systems. Big Brother is able to be integrated easily into highly bespoke web-based systems. Atterer and Albrecht~\cite{atterer2007tracking} proposed a similar interaction logging system to ours by using AJAX requests instead of web sockets. However, their events contain fewer components than ours, and it is unlikely that this older technology would scale for large user studies of today. More recently, Maxwell and Hauff~\cite{maxwell2021logui} have developed fine-grained logging infrastructure for web-based experiments that use similar modern technology to ours. While the architecture of their system may be similar to ours, we believe that the barrier to entry for logging interactions is significantly lower using Big Brother and the fact that Big Brother has been used already in a number of user studies (lab-based and embedded inside crowdsourcing services) demonstrates its robustness in different user study scenarios. We further contrast Big Brother to their LogUI system by demonstrating the additional ability to record the screen of users and the associated tools for visualising and manipulating interactions.

Lettner and Holzmann~\cite{lettner2012automated} and Kokemore and Hutter~\cite{kokemor2016aspect} both proposed toolkits for automatically logging user interactions in mobile applications. Their frameworks differ from ours as they target mobile settings (e.g., Android or iOS apps) instead of web-based interfaces. Note however that Big Brother is compatible with touch events in a mobile browser. In a similar vein, Jeong et al.~\cite{jeong2020gui} proposed a framework for visualising how users transition from one mobile application screen to the next. This system captures a screenshot each time the user transitions to a new interface in the app. This functionality is exposed in Big Brother, but this system can provide a transition graph visualisation, which we do not. 