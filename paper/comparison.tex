\section{Comparison with Existing Systems}

The idea to capture interactions in user study scenarios is not new. For example, Zuccon et al.~\cite{zuccon2013crowdsourcing} proposed a system for evaluating Information Retrieval systems using interactions from crowdsource workers. They observe interactions such as the length of sessions, the number of elements clicked, and the number of items examined. However, they do not measure fine-grained interactions such as mouse movements, and these measurements are built into the underlying interface. (unlike Big Brother which can log interactions independently of the underlying interface). Recently, Maxwell and Hauff~\cite{maxwell2021logui} have developed logging infrastructure for web-base experiments. While the architecture of their system may be similar to ours, we believe that the barrier to entry for logging interactions is significantly lower using Big Brother and that Big Brother has been `battle tested' already in a number of user studies (lab-based and embedded inside crowdsourcing services) and has been demonstrated to remain stable and record all interactions in user studies with many concurrent sessions.