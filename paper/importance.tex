%\section{Importance of System}

%Carefully designing an interface for user study experiments can be tedious and complicated. One consideration that is often overlooked, however, is tracking how users interact with a system. 
Fine-grained interaction logs enable researchers to more deeply understand how users behave in their systems. Tracking and analysing how users behave is important for a number of reasons:

\begin{description}
	\item[Evaluation] While self-reporting by users can provide a weak measure of effort required to complete tasks, tracking user interactions can provide a stronger source of evidence for how much effort, or how challenging a user found a task. For example, by logging user interactions, Cross et al.~\cite{cross2021search} found that user reported effort differed from the number of interactions between two systems.
	\item[User Modelling] \todo{todo}
	\item[Multi-Modal Interactions] Often it is desirable to analyse behavioural patterns between how users interact with the keyboard and mouse and other physical interactions, such as a users emotions~\cite{arapakis2008affective}. For example Jimmy et al.~\cite{jimmy2020health} was able to better understand what users paid attention to by combining interaction data like clicks and mouse movements with eye-tracking.
	\item[Malicious Users] By tracking how frequently a user interacts with elements on a page it is possible to identify behavioural patterns that may correspond with malicious intent~\cite{gadiraju2015understanding}. Although, there have been no studies to date that are able to detect malicious behaviour automatically from user interactions alone.
\end{description}